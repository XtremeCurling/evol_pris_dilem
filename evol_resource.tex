

\documentclass[12pt,reqno]{amsart}
\usepackage[margin=1.25 in]{geometry}

\usepackage{amssymb}
\usepackage{hyperref}
\usepackage{amsfonts}
\usepackage{epsfig, amsmath}
%\usepackage{subfigure,times}
%\usepackage{tikz,ifthen,calc}
%\usepackage{graphicx}
%\usepackage{caption}
%\usepackage[labelfont=rm]{subcaption}
%\usetikzlibrary{patterns,snakes}
%
%\newtheorem{thm}{Theorem}[section]
%\newtheorem{prop}[thm]{Proposition}
%\newtheorem{lem}[thm]{Lemma}
%\newtheorem{cor}[thm]{Corollary}
%
%\theoremstyle{definition}
%\newtheorem{definition}[thm]{Definition}
%\newtheorem{example}[thm]{Example}




%\theoremstyle{remark}
%
%\newtheorem{remark}[thm]{Remark}
%
%\numberwithin{equation}{section}
%
%\DeclareMathOperator{\dist}{dist} % The distance.

\begin{document}

\title{Evolutionary simulations with an agent-based prisoner's dilemma model}

\author{Robert Sharp}
\address{2011 Monument Ave Apt 3 $\,\,\cdot\,\,$ Richmond, VA $\,\,\cdot\,\,$ 23220}
\email{bob.wynbert.sharp@gmail.com}

%\begin{abstract}

%\end{abstract}

\maketitle

\section{Introduction}



%\begin{figure}[h]
%\begin{subfigure}{.3\textwidth}
%	\centering
%	\begin{tikzpicture}[scale=.7]
%	\draw (1, 0) -- (2, 0) -- (2, 1) -- (1, 1) -- (1, 0) -- (0, 0) -- (0, 1) -- (1, 1) -- (1, 2) -- (0, 2) -- (0, 1);
%	\draw[color=gray] (-.5, -.5) -- (2.5, 2.5);
%	\draw[color=gray] (-.5, .5) -- (1.5, 2.5);
%	\draw[color=gray] (.5, -.5) -- (2.5, 1.5);
%	\end{tikzpicture}
%	\caption{}
%	\label{fig:ribbon}
%\end{subfigure}
%\qquad
%\begin{subfigure}{.3\textwidth}
%	\centering
%	\begin{tikzpicture}[scale=.7]
%	\draw (1, 0) -- (0, 0) -- (0, 1) -- (1, 1) -- (1, 0) -- (2, 0) -- (2, 1) -- (1, 1) -- (1, 2) -- (2, 2) -- (2, 1);
%	\draw[color=gray] (-.5, -.5) -- (2.5, 2.5);
%	\draw[color=gray] (.5, -.5) -- (2.5, 1.5);
%	\end{tikzpicture}
%	\caption{}
%	\label{fig:notribbon}
%\end{subfigure}
%\caption{(\subref{fig:ribbon}) is a ribbon tromino, while (\subref{fig:notribbon}) is not}
%\label{fig:RibbonExample}
%\end{figure}



\section{The Model}

In each time period $t$, there is a finite population of $N_t$ discrete agents (we will call them ``humans'') and a continuous amount $R_t \in [0,\overline{R}]$ of the resource on which the humans feed (we will call this resource ``plants'' or ``food''). The humans are randomly matched in pairs and play a sort of prisoner's dilemma, where each can choose to exert either a low effort $e_D$ (defect) or a high effort $e_C$ (cooperate) in finding food.\footnote{Exerting low effort increases a player's chance of surviving to the next period; while finding more food also increases a player's chance of surviving. So, cooperating does not unambiguously decrease a player's chance of surviving, and thus this game isn't necessarily a true prisoner's dilemma. This will all be explained in detail later in the section.} If both players defect, their collective effort is given by $e_{DD} = 2e_D$; if one cooperates and one defects, their collective effort is given by $e_{CD} = e_C + e_D$; and if both cooperate, their collective effort is given by $e_{CC} = 2e_C + \kappa$, where $\kappa$ captures an increase in efficiency when both players work together. Each pair shares its food evenly.

This playing of the prisoner's dilemma yields $N_{CC,t}$ successful cooperators, $N_{CD,t}$ unsuccessful cooperators, $N_{DC,t} = N_{CD,t}$ successful defectors, and $N_{DD,t}$ unsuccessful defectors. Then the total effort $E_t$ exerted by the human population to find food is given by $E_t = E_{CC,t} + 2E_{CD,t} + E_{DD,t}$, where $E_{ij,t} = N_{ij,t}\cdot(e_{ij}/2)$.

Given this total effort, the human population as a whole finds and consumes $(1-e^{-\beta E_t})R_t$ of the resource for some $\beta > 0$. See Appendix \ref{resource_appendix} for the reasoning underlying this functional form. On the other hand, the plant population grows logistically, so that in period $t$, $\alpha R_t(1-R_t/\overline{R})$ new amount of plantlife is generated for some $\alpha > 0$. We normalize $\overline{R} = 1$. Then the plant population in period $t+1$ is given by
\begin{align} \label{}
R_{t+1} &= R_t + \alpha R_t(1-R_t) - R_t(1-e^{-\beta E_t}) \notag\\
	     &= R_t(\alpha(1-R_t) + e^{-\beta E_t}).
\end{align}

In order to capture decreasing returns with increased effort---i.e. the idea that spending twice as much time looking for food won't double the amount of food discovered---we don't apportion the total amount of food discovered $R_t(1-e^{-\beta E_t})$ to the groups $DD$, $DC$, $CD$ and $CC$ linearly with their corresponding total efforts $E_{ij,t}$. Instead, we imagine that the first $e_{DD}$ of effort exerted by every pair yields the same amount of food $y_{DD,t}$ for each pair; the next $e_{CD}-e_{DD}$ of effort exerted by every non-$DD$ pair yields the same amount of additional food $y_{CD,t} - y_{DD,t}$ for each of these pairs; and the remaining food is collected by way of the final $e_{CC}-e_{CD}$ of effort exerted by all mutually cooperative pairs, yielding $y_{CC,t} - y_{CD,t}$ additional food for each of these pairs.

In total, the first $e_{DD}$ of effort exerted by every pair is 
\begin{align*} \label{}
\hat{E}_{DD,t} = N_t \left(\frac{e_{DD}}{2}\right),
\end{align*}
which yields an amount of food
\begin{align*} \label{}
y_{DD,t} &= \left(\frac{1}{N_t}\right) R_t \left(1 - e^{-\beta \hat{E}_{DD,t}}\right)
\end{align*}
for each human. In total, the first $e_{DD}$ of effort exerted by every pair and $e_{CD}$ of effort exerted by every non-$DD$ pair is
\begin{align*} \label{}
\hat{E}_{CD,t} = N_t \left(\frac{e_{DD}}{2}\right) + (N_t - N_{DD,t})\left(\frac{e_{CD} - e_{DD}}{2}\right),
\end{align*}
which yields an amount of food
\begin{align*} \label{}
y_{CD,t} = y_{DC,t} = y_{DD,t} + \left(\frac{1}{N_t - N_{DD,t}}\right) R_t \cdot \left[\left(1 - e^{-\beta \hat{E}_{CD,t}}\right) - \left(1 - e^{-\beta \hat{E}_{DD,t}}\right)\right]
\end{align*}
for each human not in a $DD$ pair. The rightmost term reflects the \emph{extra} fraction of the food source discovered by the non-$DD$ pairs as a result of them exerting an additional $e_{CD} - e_{DD}$ of collective effort. Finally, as already discussed, the total sum of efforts is
\begin{align*} \label{}
\hat{E}_{CC,t} = E_t,
\end{align*}
which yields an amount of food
\begin{align*} \label{}
y_{CC,t} = y_{CD,t} + \left(\frac{1}{N_{CC,t}}\right) R_t \cdot \left[\left(1 - e^{-\beta \hat{E}_{CC,t}}\right) - \left(1 - e^{-\beta \hat{E}_{CD,t}}\right)\right]
\end{align*}
for each successfully cooperative human. Defining total consumption by all players of type $ij$ as $Y_{ij,t} = N_{ij,t} \cdot y_{ij,t}$, it is straightforward to verify that $Y_t = Y_{CC,t} + Y_{CD,t} + Y_{DC,t} + Y_{DD,t} = R_t(1-e^{-\beta E_t})$.

A player's probability of surviving to the next period is determined jointly by her effort $e$ and her consumption $y$. Higher effort increases one's chance of dying---this can be thought of as reflecting, e.g., the increased risk of a fatal accident that comes with more time spent searching for food. Formally, the effort or ``accident'' term of an individual's survival probability is $(1 - a_i)$. Since we think of effort as ``time spent searching,'' we then say that $1 - a_C = (1 - a_D)^K$, where $K = e_C/e_D$; i.e.\ if your chance of dying from an accident resulting from $e_D$ time spent searching is $a_D$, then the probability of dying from an accident resulting from $Ke_D$ time spent searching is simply $1 - (1-a_D)^K$. 

On the other hand, one's chance of surviving naturally also has a consumption term: higher consumption increases one's chance of surviving. Formally, in the model, for a player with an outcome of $ij$, this consumption or ``sustenance'' term is $(1 - e^{-\gamma y_{ij,t}})$. Putting together the accident and sustenance terms, we say a player with an outcome $ij$ has probability $\pi_{ij,t}$ of surviving to the next period, where
\begin{align} \label{surv_probability}
\pi_{ij,t} = \left(1 - a_i\right)\left(1 - e^{-\gamma y_{ij,t}}\right).
\end{align}













\newpage
\appendix

\section{Resource collection} \label{resource_appendix}

This appendix explains the mathematical reasoning underlying the equation for human resource collection: $C(E) = (1-e^{-\beta E})R$.

Consider that there is a number $A$ of discrete areas to explore, each of which either has food or does not; say $R$ of these areas have food and $A - R$ do not. Consider also that a group of humans explores areas at random $E$ times; that is, this group of humans makes $E$ collective random ``draws'' with replacement from the set of $A$ areas, so that with each draw, the chance of finding a specific area is $1/A$. Then how many unique areas do we expect these humans to collectively explore---and as a result of this, how many units of food do we expect them to find?

The probability that a given area is \emph{not} explored is given by $((A-1)/A)^E$, so that the probability that this area \emph{is} explored is $1 - ((A-1)/A)^E$. Then using an indicator function, we have that
\begin{align*} \label{}
\mathbb{E}[\textrm{\# unique areas discovered}] &= \sum_{i=1}^A\left(1-\left(\frac{A-1}{A}\right)^E\right) \\
									&= A\left(1-\left(\frac{A-1}{A}\right)^E\right).
\end{align*}
Then the expected fraction of the food source discovered is simply
\begin{align*} \label{}
\mathbb{E}[\textrm{fraction of existing food discovered}] &= \frac{\left(\frac{F}{A}\right)A\left(1-\left(\frac{A-1}{A}\right)^E\right)}{F} \\
									&= 1-\left(\frac{A-1}{A}\right)^E.
\end{align*}

Now, we want to examine what fraction of the existing food source is consumed in a continuous region with a continuous food source; so, we examine the limit of the above expectation as $A \rightarrow \infty$, holding $F/A = \overline{F}$ and $E/A = \overline{E}$ constant ($\overline{F}$ corresponds to our measure $R$ of the resource, while $\overline{E}$ corresponds to our measure $E$ of collective effort). In particular, we want to find a closed-form expression for $\lim_{A \rightarrow \infty} (1-((A-1)/A)^{\overline{E}A})$. This limit is evaluated below:
\begin{align*} \label{}
\lim_{A\rightarrow\infty} \left(1 - \left(\frac{A-1}{A}\right)^{\overline{E}A}\right) &= 1 - \left(\lim_{A\rightarrow\infty} \left(\frac{A-1}{A}\right)^A\right)^{\overline{E}};
\end{align*}
so, we need to evaluate $\lim_{A\rightarrow\infty} ((A-1)/A)^A$. By the continuity of the limit, we know that
\begin{align*} \label{}
\ln\left(\lim_{A\rightarrow\infty} ((A-1)/A)^A\right) &= \lim_{A\rightarrow\infty}\left(\ln\left(((A-1)/A)^A\right)\right) \\
									&= \lim_{A\rightarrow\infty}\frac{\ln(A-1) - \ln(A)}{1/A} \\
									&= \lim_{A\rightarrow\infty}\frac{1/(A-1) - 1/A}{-1/A^2} \\
									&= \lim_{A\rightarrow\infty}\frac{-A^2}{A^2 - A} \\
									&= -1,
\end{align*}
where the third equality is an application of L'Hopital's rule, which applies because the limits of the numerator and denominator in the expression on the second line are both zero.

So,
\begin{align*} \label{}
\lim_{A\rightarrow\infty} \left(\frac{A-1}{A}\right)^A &= e^{-1},
\end{align*}
so that
\begin{align*} \label{}
\mathbb{E}[\textrm{fraction of existing food discovered}] &= 1 - \left(\lim_{A\rightarrow\infty} \left(\frac{A-1}{A}\right)^A\right)^{\overline{E}} \\
										     &= 1 - e^{-\overline{E}}.
\end{align*}

So, it makes sense to follow this form when modeling human food discovery (i.e.\ collection, i.e.\ consumption) in the model. But, since collective effort $E$ is not normalized specifically by any notion of area, there is an extra degree of freedom; hence, we can reasonably introduce the factor $\beta$. Then in the model, the fraction of food discovered is $1 - e^{-\beta E}$, and thus the level of food discovered is simply $C(E) = (1 - e^{-\beta E})R$.







%\bibliographystyle{ieeetr}
%\bibliography{cl,pakribbons,reidhomotopy}
%\vspace*{.7cm}









\end{document}
